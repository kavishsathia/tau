
\documentclass[a4paper]{article}

\usepackage[english]{babel}
\usepackage[utf8x]{inputenc}
\usepackage[active,tightpage]{preview}
\usepackage{amsmath}
\usepackage{graphicx}
\usepackage[colorinlistoftodos]{todonotes}

\title{Jess's mathematics}
\author{Jessies}

\begin{document}
\begin{preview}
\maketitle

\begin{abstract}
\LaTeX{} is the best way to write mathematics. It completely pisses all over Word. However, it does take some time to get used to so might not be worth your while if you won't write too much. The way I use it is to first download and install a latex editor and then get writing, but I would recommend that you use this website instead since you can get going a lot quicker. The upshot of the whole business is that you type in here and then a pdf is generated with all the equations looking ace. I'll give you some examples. 
\end{abstract}

\section{Writing some mathematics}

\LaTeX{} is great at typesetting mathematics. Let's say you want to write some equations involving integrals. You have to first setup and ``equation environment'' and then plug in the correct commands for the integral signs etc. An example is the following.
\begin{equation}
		y(s) = \int_0^1 \left(t^2+\ln(t)+\frac{1}{t}\right)\text{d}t.
\end{equation}
Here I've used a the command text to get the straight d in the integral which is usually used to denote the infinitesimal in integration, but this doesn't really matter. 

How about if you want to write a differential equation. 
\begin{equation}
		\frac{\text{d}^2y}{\text{d}x^2}+ x^3\frac{\text{d}y}{\text{d}x}+y = e^{3x}.
\end{equation}
You can see that the frac command is pretty handy here.

If you want to learn more. Try googling ``Latex for beginners'' or something and then taking it from there. It requires an investment of time to get used to Latex but is really worth it in the longrun if you are going to write lots of mathematics. Every scientist and engineer in the world uses this software so it's not just small peanuts!

Anyway, hope this is helpful and that the tone isn't too patronising!

Sam

\end{preview}
\end{document}